\addcontentsline{toc}{chapter}{Abstract}



\begin{center}
\textbf{\large Abstract}
\end{center}


\emph{``High intensity focused ultrasound treatment is a potential non-invasive treatment which uses ultrasound energy generated by a piezoelectric (ferroelectric) transducer to thermally ablate tumour tissues. However, the issue that is preventing the treatment from being more widely available is a prominent treatment cost in comparison to alternative therapies. The high treatment cost can be attributed to the longer operation time with cycles of heating and cooling. This heating and cooling cycle is applied in order to prevent overheating of the ferroelectric material which is used to generate the ultrasound. Overheating of the transducer can change the effective thickness frequency relationship of the material and can even lead to depolarization of the material. The overheating is caused by the energy loss (dielectric dissipation), which occurs when the alternating electric field is applied and converted into the ultrasound. The associated material characteristics are the quality factor or tan , which is a macroscopic property. Alternatively, the loss can also be related to the area of the hysteresis loop of the particular material. This project aims at searching for potential ferroelectric materials with reduced overheating which in turn means a material with reduced hysteresis area (or low tan ). At the initial stage, first principle approaches have been adapted in our research rather than experimental methods, which would consume more effort in terms of equipment, capital and time.  For the purpose of study, all electron density functional package �WIEN2k� with the help of high performance computing is being used. In order to determine the ferroelectric parameters which are related to the polarization based property of materials, an additional software �BerryPI� has been developed in the framework of our research. The property of interest (tan? or hysteresis area) which is measured at macroscopic level has been brought down to microscopic level where it has been related to the barrier height of the potential curve of the ferroelectric materials. Based on this relation our plan is to perform screening of potential ferroelectric materials aiming at optimizing the produced mechanical energy to power loss ratio.''}

